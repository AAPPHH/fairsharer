\documentclass[a4paper,12pt]{article}

\usepackage{hyperref}

\title{GitHub Repository für Fingerübung 5 - Testing!}
\date{4. Dezember 2024}

\begin{document}

\maketitle

\section*{Link zum Repository}

Hier ist der Link zu dem GitHub-Repository, das die implementierte Funktion, die Testfunktion und den funktionierenden Workflow enthält:

\noindent
\href{https://github.com/AAPPHH/fairsharer}{https://github.com/AAPPHH/fairsharer}

\section*{Inhalt des Repositories}

Das Repository enthält:
\begin{itemize}
    \item Die Python-Funktion \texttt{fair\_sharer}.
    \item Eine Testdatei \texttt{test\_fairsharer.py}, die Unit Tests für die Funktion enthält.
    \item Einen GitHub Actions Workflow zur automatisierten Überprüfung des Codes.
    \item Eine \texttt{requirements.txt}-Datei mit allen benötigten Abhängigkeiten.
\end{itemize}

\end{document}
